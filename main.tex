\documentclass{article}

\usepackage[utf8]{inputenc}
\usepackage{graphicx}		% Graphics.
\usepackage{color}
\usepackage[english]{babel}
\usepackage{float}
\usepackage{subcaption}
\usepackage{xfrac}
\usepackage{matlab-prettifier}
\usepackage{amsmath}    
\usepackage{amssymb}
\usepackage{siunitx}
\usepackage{pdfpages}
\usepackage{hyperref}

% Create a separate table for the appendix.
\usepackage[toc,page]{appendix}

% Table of Content has fast links to sections.
\usepackage{hyperref}

% Remove dots in table of contents.
\usepackage[titles]{tocloft}
\renewcommand{\cftdot}{}

% Page style.
\usepackage[top=2cm, bottom=2cm, left = 2cm, right = 2cm]{geometry}
\setlength{\parindent}{0pt}	% Disable indents.

\begin{document}

%----------------------------------------------------------------------------------------
%	Title page.
%----------------------------------------------------------------------------------------
\input{chapters/titlepage.tex}


%----------------------------------------------------------------------------------------
%	TABLE OF CONTENT.
%----------------------------------------------------------------------------------------
\newpage				% Start at new page.
\pagenumbering{arabic}	% Page numbering reset & style.
\renewcommand{\contentsname}{Table of Contents}
\tableofcontents		% Add table of content.


%----------------------------------------------------------------------------------------
%	QUESTION 1.
%----------------------------------------------------------------------------------------
\newpage
\section{Find Station No. 03808. Which location is this station attributed to? Where is it situated?}
As can be seen from figure \ref{fig:1}, the grounstation 03808 is located in south west part of England, more specifically in Camborne. 

\begin{figure}[H]
	\centering
 	\includegraphics[width=1\textwidth]{figures/europe.jpg}
 	\caption{Europa weather stations.}
 	\label{fig:1}
\end{figure}


%----------------------------------------------------------------------------------------
%	QUESTION 2.
%----------------------------------------------------------------------------------------
\section{Extract the radiosonde profile data for 2007-07-05 0:00 UTC, 12:00 UTC and 2007-07-06 0:00 UTC}
% http://weather.uwyo.edu/cgi-bin/sounding?region=europe&TYPE=TEXT%3ALIST&YEAR=2007&MONTH=07&FROM=0500&TO=0600&STNM=03808
The information can be found online by clicking on the following URL:\\
\url{http://weather.uwyo.edu/cgi-bin/sounding?region=europe\&TYPE=TEXT%3ALIST&YEAR=2007&MONTH=07&FROM=0500&TO=0600&STNM=03808}\\

The information can also be found on the next page.
\includepdf[pages=-]{figures/Radiosonde-Data.pdf}


%----------------------------------------------------------------------------------------
%	QUESTION 3.
%----------------------------------------------------------------------------------------
\newpage
\section{Plot the profiles of temperature T, dewpoint temperature $T_D$ and relative humidity RH for each of theses dates. (T and $T_D$ in one plot, RH in a separate one) Use a suitable altitude coordinate!}



%----------------------------------------------------------------------------------------
%	QUESTION 4.
%----------------------------------------------------------------------------------------
\newpage
\section{How are T and $T_D$ related to each other? Consider RH for your argumentation}
The dew point temperature ($T_D$) is the temperature of which the air must be cooled to to become saturated with water vapour. This is dependent on the current temperature and the relative humidity (RH). When the temperature increases, it can hold more water vapour and thus the dew point temperature increases for the same relative humidity (RH). When the relative humidity increases, under the same temperature, then the dew point temperature goes up as well.\\

When cooled below the dew point temperature, the airborne water vapor will condense to form liquid water (dew). When air cools to its dew point through contact with a surface that is colder than the air, water will condense on the surface.


%----------------------------------------------------------------------------------------
%	QUESTION 5.
%----------------------------------------------------------------------------------------
\newpage
\section{At which altitude for these data is the tropopause height located?What method do you use to determine the tropopause?}


%----------------------------------------------------------------------------------------
%	QUESTION 6.
%----------------------------------------------------------------------------------------
\newpage
\section{Describe the some crucial differences between the tropospheric parts of the above plotted profiles. What did probably happen during that date?}



%----------------------------------------------------------------------------------------
%	QUESTION 7.
%----------------------------------------------------------------------------------------
\newpage
\section{Plot the data for 2007-07-05, 12:00 UTC as Stuve diagram. What does the Stuve diagram show? What do the different “help lines” mean. What is this type of diagram used for? Use internet resources to find out}


\end{document}
